Handling large data files in R: data.table and more!

Run the following code in a bash or mac terminal to obtain a copy of this repository:

`git clone git@github.com:TomKellyGenetics/syska-R-data-table.git`

Lesson developed using the gapminder dataset (as used in Software Carpentry and Research Bazaar), data and examples of ggplot or pylr from this repository https://github.com/resbaz/r-novice-gapminder

Materials developed at the University of Otago (Biochemistry department) by postgraduate students. This lesson can be followed with the commented R script, Rmarkdown document, or any preferred document produced by `knitr`.

Uses a number of R packages in read/write files with benchmarking on file i/o and a comparison of data.table class to the data.frame on which it is based (and most R users are familiar with). Most of these packages can be installed on the fly on CRAN if network access is avaiable during the lesson, due to large dependencies we recommend install `devtools` and `feather` in advance along these features are only discussed in terms of benchmarking similar functionality in other packages. Large data files are used in the lesson to demonstrate the power and speed of various features, however these were not included the the repository on github due to limiting speed for lesson participants to clone the repository during the lesson. These large data files are constructed with rbind of many copies of the dataset used in the exercises, this demonstrates the power of these packages to deal with data files up to the scales of mega- or giga-bytes which are frequently encountered in Genomics or Bioinformatics analysis. 
\documentclass[]{article}
\usepackage{lmodern}
\usepackage{amssymb,amsmath}
\usepackage{ifxetex,ifluatex}
\usepackage{fixltx2e} % provides \textsubscript
\ifnum 0\ifxetex 1\fi\ifluatex 1\fi=0 % if pdftex
  \usepackage[T1]{fontenc}
  \usepackage[utf8]{inputenc}
\else % if luatex or xelatex
  \ifxetex
    \usepackage{mathspec}
    \usepackage{xltxtra,xunicode}
  \else
    \usepackage{fontspec}
  \fi
  \defaultfontfeatures{Mapping=tex-text,Scale=MatchLowercase}
  \newcommand{\euro}{€}
\fi
% use upquote if available, for straight quotes in verbatim environments
\IfFileExists{upquote.sty}{\usepackage{upquote}}{}
% use microtype if available
\IfFileExists{microtype.sty}{%
\usepackage{microtype}
\UseMicrotypeSet[protrusion]{basicmath} % disable protrusion for tt fonts
}{}
\usepackage[margin=1in]{geometry}
\ifxetex
  \usepackage[setpagesize=false, % page size defined by xetex
              unicode=false, % unicode breaks when used with xetex
              xetex]{hyperref}
\else
  \usepackage[unicode=true]{hyperref}
\fi
\hypersetup{breaklinks=true,
            bookmarks=true,
            pdfauthor={},
            pdftitle={},
            colorlinks=true,
            citecolor=blue,
            urlcolor=blue,
            linkcolor=magenta,
            pdfborder={0 0 0}}
\urlstyle{same}  % don't use monospace font for urls
\usepackage{color}
\usepackage{fancyvrb}
\newcommand{\VerbBar}{|}
\newcommand{\VERB}{\Verb[commandchars=\\\{\}]}
\DefineVerbatimEnvironment{Highlighting}{Verbatim}{commandchars=\\\{\}}
% Add ',fontsize=\small' for more characters per line
\usepackage{framed}
\definecolor{shadecolor}{RGB}{248,248,248}
\newenvironment{Shaded}{\begin{snugshade}}{\end{snugshade}}
\newcommand{\KeywordTok}[1]{\textcolor[rgb]{0.13,0.29,0.53}{\textbf{{#1}}}}
\newcommand{\DataTypeTok}[1]{\textcolor[rgb]{0.13,0.29,0.53}{{#1}}}
\newcommand{\DecValTok}[1]{\textcolor[rgb]{0.00,0.00,0.81}{{#1}}}
\newcommand{\BaseNTok}[1]{\textcolor[rgb]{0.00,0.00,0.81}{{#1}}}
\newcommand{\FloatTok}[1]{\textcolor[rgb]{0.00,0.00,0.81}{{#1}}}
\newcommand{\ConstantTok}[1]{\textcolor[rgb]{0.00,0.00,0.00}{{#1}}}
\newcommand{\CharTok}[1]{\textcolor[rgb]{0.31,0.60,0.02}{{#1}}}
\newcommand{\SpecialCharTok}[1]{\textcolor[rgb]{0.00,0.00,0.00}{{#1}}}
\newcommand{\StringTok}[1]{\textcolor[rgb]{0.31,0.60,0.02}{{#1}}}
\newcommand{\VerbatimStringTok}[1]{\textcolor[rgb]{0.31,0.60,0.02}{{#1}}}
\newcommand{\SpecialStringTok}[1]{\textcolor[rgb]{0.31,0.60,0.02}{{#1}}}
\newcommand{\ImportTok}[1]{{#1}}
\newcommand{\CommentTok}[1]{\textcolor[rgb]{0.56,0.35,0.01}{\textit{{#1}}}}
\newcommand{\DocumentationTok}[1]{\textcolor[rgb]{0.56,0.35,0.01}{\textbf{\textit{{#1}}}}}
\newcommand{\AnnotationTok}[1]{\textcolor[rgb]{0.56,0.35,0.01}{\textbf{\textit{{#1}}}}}
\newcommand{\CommentVarTok}[1]{\textcolor[rgb]{0.56,0.35,0.01}{\textbf{\textit{{#1}}}}}
\newcommand{\OtherTok}[1]{\textcolor[rgb]{0.56,0.35,0.01}{{#1}}}
\newcommand{\FunctionTok}[1]{\textcolor[rgb]{0.00,0.00,0.00}{{#1}}}
\newcommand{\VariableTok}[1]{\textcolor[rgb]{0.00,0.00,0.00}{{#1}}}
\newcommand{\ControlFlowTok}[1]{\textcolor[rgb]{0.13,0.29,0.53}{\textbf{{#1}}}}
\newcommand{\OperatorTok}[1]{\textcolor[rgb]{0.81,0.36,0.00}{\textbf{{#1}}}}
\newcommand{\BuiltInTok}[1]{{#1}}
\newcommand{\ExtensionTok}[1]{{#1}}
\newcommand{\PreprocessorTok}[1]{\textcolor[rgb]{0.56,0.35,0.01}{\textit{{#1}}}}
\newcommand{\AttributeTok}[1]{\textcolor[rgb]{0.77,0.63,0.00}{{#1}}}
\newcommand{\RegionMarkerTok}[1]{{#1}}
\newcommand{\InformationTok}[1]{\textcolor[rgb]{0.56,0.35,0.01}{\textbf{\textit{{#1}}}}}
\newcommand{\WarningTok}[1]{\textcolor[rgb]{0.56,0.35,0.01}{\textbf{\textit{{#1}}}}}
\newcommand{\AlertTok}[1]{\textcolor[rgb]{0.94,0.16,0.16}{{#1}}}
\newcommand{\ErrorTok}[1]{\textcolor[rgb]{0.64,0.00,0.00}{\textbf{{#1}}}}
\newcommand{\NormalTok}[1]{{#1}}
\usepackage{longtable,booktabs}
\usepackage{graphicx,grffile}
\makeatletter
\def\maxwidth{\ifdim\Gin@nat@width>\linewidth\linewidth\else\Gin@nat@width\fi}
\def\maxheight{\ifdim\Gin@nat@height>\textheight\textheight\else\Gin@nat@height\fi}
\makeatother
% Scale images if necessary, so that they will not overflow the page
% margins by default, and it is still possible to overwrite the defaults
% using explicit options in \includegraphics[width, height, ...]{}
\setkeys{Gin}{width=\maxwidth,height=\maxheight,keepaspectratio}
\setlength{\parindent}{0pt}
\setlength{\parskip}{6pt plus 2pt minus 1pt}
\setlength{\emergencystretch}{3em}  % prevent overfull lines
\providecommand{\tightlist}{%
  \setlength{\itemsep}{0pt}\setlength{\parskip}{0pt}}
\setcounter{secnumdepth}{0}

%%% Use protect on footnotes to avoid problems with footnotes in titles
\let\rmarkdownfootnote\footnote%
\def\footnote{\protect\rmarkdownfootnote}

%%% Change title format to be more compact
\usepackage{titling}

% Create subtitle command for use in maketitle
\newcommand{\subtitle}[1]{
  \posttitle{
    \begin{center}\large#1\end{center}
    }
}

\setlength{\droptitle}{-2em}
  \title{}
  \pretitle{\vspace{\droptitle}}
  \posttitle{}
  \author{}
  \preauthor{}\postauthor{}
  \date{}
  \predate{}\postdate{}


% Redefines (sub)paragraphs to behave more like sections
\ifx\paragraph\undefined\else
\let\oldparagraph\paragraph
\renewcommand{\paragraph}[1]{\oldparagraph{#1}\mbox{}}
\fi
\ifx\subparagraph\undefined\else
\let\oldsubparagraph\subparagraph
\renewcommand{\subparagraph}[1]{\oldsubparagraph{#1}\mbox{}}
\fi

\begin{document}
\maketitle

\section{Stuff You Should Know About: Handling Large Data Files with
R}\label{stuff-you-should-know-about-handling-large-data-files-with-r}

\subsection{The Data Table package and various other ways to handle data
in
R}\label{the-data-table-package-and-various-other-ways-to-handle-data-in-r}

\begin{itemize}
\tightlist
\item
  \textbf{Authors}: Tom Kelly
\item
  \textbf{Research field}: Bioinformatics / Computational Biology /
  Cancer Genomics
\item
  \textbf{Lesson Topic}: An introduction to various packages for file
  I/O and data manipulation in R, with comparision to base R (and
  compatibility with data frames), in terms of user-friendliness,
  performance in CPU-time, and memory usage.
\end{itemize}

\subsection{Installation}\label{installation}

Install Data Table from CRAN (current version 1.9.6)

\begin{Shaded}
\begin{Highlighting}[]
\KeywordTok{install.packages}\NormalTok{(}\StringTok{"data.table"}\NormalTok{, }\DataTypeTok{repos =} \StringTok{"https::/cran.rstudio.com"}\NormalTok{)}
\KeywordTok{library}\NormalTok{(}\StringTok{"data.table"}\NormalTok{)}
\end{Highlighting}
\end{Shaded}

Install development version from GitHub (current version 1.9.7)

\begin{Shaded}
\begin{Highlighting}[]
\KeywordTok{install.packages}\NormalTok{(}\StringTok{"data.table"}\NormalTok{, }\DataTypeTok{repos =} \StringTok{"https://Rdatatable.github.io/data.table"}\NormalTok{, }\DataTypeTok{type =} \StringTok{"source"}\NormalTok{) }\CommentTok{#v1.9.7}
\KeywordTok{library}\NormalTok{(}\StringTok{"data.table"}\NormalTok{)}
\end{Highlighting}
\end{Shaded}

\subsection{Getting Started: Data
Frames}\label{getting-started-data-frames}

data table has it's own read function - to rapidly read data into R
Backwards compatible: It can be used for data.frames

\begin{Shaded}
\begin{Highlighting}[]
\NormalTok{gapminderFiveYearData <-}\StringTok{ }\KeywordTok{fread}\NormalTok{(}\StringTok{"gapminder-FiveYearData.csv"}\NormalTok{, }\DataTypeTok{data.table=}\NormalTok{F)}
\KeywordTok{class}\NormalTok{(gapminderFiveYearData)}
\end{Highlighting}
\end{Shaded}

\begin{verbatim}
## [1] "data.frame"
\end{verbatim}

\begin{Shaded}
\begin{Highlighting}[]
\KeywordTok{dim}\NormalTok{(gapminderFiveYearData)}
\end{Highlighting}
\end{Shaded}

\begin{verbatim}
## [1] 1704    6
\end{verbatim}

\begin{Shaded}
\begin{Highlighting}[]
\KeywordTok{head}\NormalTok{(gapminderFiveYearData)}
\end{Highlighting}
\end{Shaded}

\begin{verbatim}
##       country year      pop continent lifeExp gdpPercap
## 1 Afghanistan 1952  8425333      Asia  28.801  779.4453
## 2 Afghanistan 1957  9240934      Asia  30.332  820.8530
## 3 Afghanistan 1962 10267083      Asia  31.997  853.1007
## 4 Afghanistan 1967 11537966      Asia  34.020  836.1971
## 5 Afghanistan 1972 13079460      Asia  36.088  739.9811
## 6 Afghanistan 1977 14880372      Asia  38.438  786.1134
\end{verbatim}

\begin{Shaded}
\begin{Highlighting}[]
\KeywordTok{tail}\NormalTok{(gapminderFiveYearData)}
\end{Highlighting}
\end{Shaded}

\begin{verbatim}
##       country year      pop continent lifeExp gdpPercap
## 1699 Zimbabwe 1982  7636524    Africa  60.363  788.8550
## 1700 Zimbabwe 1987  9216418    Africa  62.351  706.1573
## 1701 Zimbabwe 1992 10704340    Africa  60.377  693.4208
## 1702 Zimbabwe 1997 11404948    Africa  46.809  792.4500
## 1703 Zimbabwe 2002 11926563    Africa  39.989  672.0386
## 1704 Zimbabwe 2007 12311143    Africa  43.487  469.7093
\end{verbatim}

\begin{Shaded}
\begin{Highlighting}[]
\KeywordTok{str}\NormalTok{(gapminderFiveYearData)}
\end{Highlighting}
\end{Shaded}

\begin{verbatim}
## 'data.frame':    1704 obs. of  6 variables:
##  $ country  : chr  "Afghanistan" "Afghanistan" "Afghanistan" "Afghanistan" ...
##  $ year     : int  1952 1957 1962 1967 1972 1977 1982 1987 1992 1997 ...
##  $ pop      : num  8425333 9240934 10267083 11537966 13079460 ...
##  $ continent: chr  "Asia" "Asia" "Asia" "Asia" ...
##  $ lifeExp  : num  28.8 30.3 32 34 36.1 ...
##  $ gdpPercap: num  779 821 853 836 740 ...
\end{verbatim}

Backwards compatible: these are standard dataframes compatible with
ggplots

\begin{Shaded}
\begin{Highlighting}[]
\KeywordTok{library}\NormalTok{(}\StringTok{"ggplot2"}\NormalTok{)}
\KeywordTok{ggplot}\NormalTok{(}\DataTypeTok{data =} \NormalTok{gapminderFiveYearData, }\KeywordTok{aes}\NormalTok{(}\DataTypeTok{x =} \NormalTok{lifeExp, }\DataTypeTok{y =} \NormalTok{gdpPercap, }\DataTypeTok{color=}\NormalTok{continent)) +}
\StringTok{  }\KeywordTok{geom_point}\NormalTok{()}
\end{Highlighting}
\end{Shaded}

\includegraphics{Figs/unnamed-chunk-4-1.pdf}\\
 \#\#Introducing Data Tables

data table defaults to reading it's own data.table format

\begin{Shaded}
\begin{Highlighting}[]
\NormalTok{gapminderFiveYearData <-}\StringTok{ }\KeywordTok{fread}\NormalTok{(}\StringTok{"gapminder-FiveYearData.csv"}\NormalTok{)}
\KeywordTok{class}\NormalTok{(gapminderFiveYearData)}
\end{Highlighting}
\end{Shaded}

\begin{verbatim}
## [1] "data.table" "data.frame"
\end{verbatim}

\begin{Shaded}
\begin{Highlighting}[]
\KeywordTok{dim}\NormalTok{(gapminderFiveYearData)}
\end{Highlighting}
\end{Shaded}

\begin{verbatim}
## [1] 1704    6
\end{verbatim}

\begin{Shaded}
\begin{Highlighting}[]
\KeywordTok{head}\NormalTok{(gapminderFiveYearData)}
\end{Highlighting}
\end{Shaded}

\begin{verbatim}
##        country year      pop continent lifeExp gdpPercap
## 1: Afghanistan 1952  8425333      Asia  28.801  779.4453
## 2: Afghanistan 1957  9240934      Asia  30.332  820.8530
## 3: Afghanistan 1962 10267083      Asia  31.997  853.1007
## 4: Afghanistan 1967 11537966      Asia  34.020  836.1971
## 5: Afghanistan 1972 13079460      Asia  36.088  739.9811
## 6: Afghanistan 1977 14880372      Asia  38.438  786.1134
\end{verbatim}

\begin{Shaded}
\begin{Highlighting}[]
\KeywordTok{tail}\NormalTok{(gapminderFiveYearData)}
\end{Highlighting}
\end{Shaded}

\begin{verbatim}
##     country year      pop continent lifeExp gdpPercap
## 1: Zimbabwe 1982  7636524    Africa  60.363  788.8550
## 2: Zimbabwe 1987  9216418    Africa  62.351  706.1573
## 3: Zimbabwe 1992 10704340    Africa  60.377  693.4208
## 4: Zimbabwe 1997 11404948    Africa  46.809  792.4500
## 5: Zimbabwe 2002 11926563    Africa  39.989  672.0386
## 6: Zimbabwe 2007 12311143    Africa  43.487  469.7093
\end{verbatim}

\begin{Shaded}
\begin{Highlighting}[]
\KeywordTok{str}\NormalTok{(gapminderFiveYearData)}
\end{Highlighting}
\end{Shaded}

\begin{verbatim}
## Classes 'data.table' and 'data.frame':   1704 obs. of  6 variables:
##  $ country  : chr  "Afghanistan" "Afghanistan" "Afghanistan" "Afghanistan" ...
##  $ year     : int  1952 1957 1962 1967 1972 1977 1982 1987 1992 1997 ...
##  $ pop      : num  8425333 9240934 10267083 11537966 13079460 ...
##  $ continent: chr  "Asia" "Asia" "Asia" "Asia" ...
##  $ lifeExp  : num  28.8 30.3 32 34 36.1 ...
##  $ gdpPercap: num  779 821 853 836 740 ...
##  - attr(*, ".internal.selfref")=<externalptr>
\end{verbatim}

Data tables also auto-trim when printing to console

\begin{Shaded}
\begin{Highlighting}[]
\NormalTok{gapminderFiveYearData}
\end{Highlighting}
\end{Shaded}

\begin{verbatim}
##           country year      pop continent lifeExp gdpPercap
##    1: Afghanistan 1952  8425333      Asia  28.801  779.4453
##    2: Afghanistan 1957  9240934      Asia  30.332  820.8530
##    3: Afghanistan 1962 10267083      Asia  31.997  853.1007
##    4: Afghanistan 1967 11537966      Asia  34.020  836.1971
##    5: Afghanistan 1972 13079460      Asia  36.088  739.9811
##   ---                                                      
## 1700:    Zimbabwe 1987  9216418    Africa  62.351  706.1573
## 1701:    Zimbabwe 1992 10704340    Africa  60.377  693.4208
## 1702:    Zimbabwe 1997 11404948    Africa  46.809  792.4500
## 1703:    Zimbabwe 2002 11926563    Africa  39.989  672.0386
## 1704:    Zimbabwe 2007 12311143    Africa  43.487  469.7093
\end{verbatim}

data tables are backwards compatible with a lot of operations which use
data.frames Such as plots\ldots{}

\begin{Shaded}
\begin{Highlighting}[]
\KeywordTok{ggplot}\NormalTok{(}\DataTypeTok{data =} \NormalTok{gapminderFiveYearData, }\KeywordTok{aes}\NormalTok{(}\DataTypeTok{x =} \NormalTok{lifeExp, }\DataTypeTok{y =} \NormalTok{gdpPercap, }\DataTypeTok{color=}\NormalTok{continent)) +}
\StringTok{  }\KeywordTok{geom_point}\NormalTok{()}
\end{Highlighting}
\end{Shaded}

\includegraphics{Figs/unnamed-chunk-7-1.pdf}\\
\ldots{} and linear models\ldots{}

\begin{Shaded}
\begin{Highlighting}[]
\NormalTok{linear_model <-}\StringTok{ }\KeywordTok{lm}\NormalTok{(gdpPercap ~}\StringTok{ }\NormalTok{pop +}\StringTok{ }\NormalTok{year, gapminderFiveYearData)}
\KeywordTok{summary}\NormalTok{(linear_model)}
\end{Highlighting}
\end{Shaded}

\begin{verbatim}
## 
## Call:
## lm(formula = gdpPercap ~ pop + year, data = gapminderFiveYearData)
## 
## Residuals:
##    Min     1Q Median     3Q    Max 
## -10537  -5356  -2811   2043 109153 
## 
## Coefficients:
##               Estimate Std. Error t value Pr(>|t|)    
## (Intercept) -2.537e+05  2.674e+04  -9.487   <2e-16 ***
## pop         -4.143e-06  2.198e-06  -1.885   0.0596 .  
## year         1.319e+02  1.351e+01   9.760   <2e-16 ***
## ---
## Signif. codes:  0 '***' 0.001 '**' 0.01 '*' 0.05 '.' 0.1 ' ' 1
## 
## Residual standard error: 9595 on 1701 degrees of freedom
## Multiple R-squared:  0.05365,    Adjusted R-squared:  0.05254 
## F-statistic: 48.22 on 2 and 1701 DF,  p-value: < 2.2e-16
\end{verbatim}

\begin{Shaded}
\begin{Highlighting}[]
\NormalTok{linear_model <-}\StringTok{ }\KeywordTok{lm}\NormalTok{(lifeExp ~}\StringTok{ }\NormalTok{gdpPercap +}\StringTok{ }\NormalTok{pop +}\StringTok{ }\NormalTok{year, gapminderFiveYearData)}
\KeywordTok{summary}\NormalTok{(linear_model)}
\end{Highlighting}
\end{Shaded}

\begin{verbatim}
## 
## Call:
## lm(formula = lifeExp ~ gdpPercap + pop + year, data = gapminderFiveYearData)
## 
## Residuals:
##     Min      1Q  Median      3Q     Max 
## -67.497  -7.075   1.121   7.701  19.640 
## 
## Coefficients:
##               Estimate Std. Error t value Pr(>|t|)    
## (Intercept) -4.115e+02  2.767e+01 -14.872  < 2e-16 ***
## gdpPercap    6.729e-04  2.444e-05  27.529  < 2e-16 ***
## pop          6.353e-09  2.218e-09   2.864  0.00423 ** 
## year         2.354e-01  1.400e-02  16.812  < 2e-16 ***
## ---
## Signif. codes:  0 '***' 0.001 '**' 0.01 '*' 0.05 '.' 0.1 ' ' 1
## 
## Residual standard error: 9.673 on 1700 degrees of freedom
## Multiple R-squared:  0.4402, Adjusted R-squared:  0.4392 
## F-statistic: 445.6 on 3 and 1700 DF,  p-value: < 2.2e-16
\end{verbatim}

\begin{Shaded}
\begin{Highlighting}[]
\NormalTok{linear_model <-}\StringTok{ }\KeywordTok{glm}\NormalTok{(lifeExp ~}\StringTok{ }\NormalTok{gdpPercap +}\StringTok{ }\NormalTok{continent +}\StringTok{ }\NormalTok{pop +}\StringTok{ }\NormalTok{year, }\DataTypeTok{family  =}\StringTok{"gaussian"}\NormalTok{, gapminderFiveYearData)}
\KeywordTok{summary}\NormalTok{(linear_model)}
\end{Highlighting}
\end{Shaded}

\begin{verbatim}
## 
## Call:
## glm(formula = lifeExp ~ gdpPercap + continent + pop + year, family = "gaussian", 
##     data = gapminderFiveYearData)
## 
## Deviance Residuals: 
##      Min        1Q    Median        3Q       Max  
## -28.4051   -4.0550    0.2317    4.5073   20.0217  
## 
## Coefficients:
##                     Estimate Std. Error t value Pr(>|t|)    
## (Intercept)       -5.185e+02  1.989e+01 -26.062   <2e-16 ***
## gdpPercap          2.985e-04  2.002e-05  14.908   <2e-16 ***
## continentAmericas  1.429e+01  4.946e-01  28.898   <2e-16 ***
## continentAsia      9.375e+00  4.719e-01  19.869   <2e-16 ***
## continentEurope    1.936e+01  5.182e-01  37.361   <2e-16 ***
## continentOceania   2.056e+01  1.469e+00  13.995   <2e-16 ***
## pop                1.791e-09  1.634e-09   1.096    0.273    
## year               2.863e-01  1.006e-02  28.469   <2e-16 ***
## ---
## Signif. codes:  0 '***' 0.001 '**' 0.01 '*' 0.05 '.' 0.1 ' ' 1
## 
## (Dispersion parameter for gaussian family taken to be 47.37935)
## 
##     Null deviance: 284148  on 1703  degrees of freedom
## Residual deviance:  80355  on 1696  degrees of freedom
## AIC: 11420
## 
## Number of Fisher Scoring iterations: 2
\end{verbatim}

\ldots{} and data manipulation packages (plyr, dplyr, reshape, tidyr,
etc\ldots{})

\begin{Shaded}
\begin{Highlighting}[]
\KeywordTok{library}\NormalTok{(}\StringTok{"plyr"}\NormalTok{)}
\NormalTok{calcGDP <-}\StringTok{ }\NormalTok{function(dat, }\DataTypeTok{year=}\OtherTok{NULL}\NormalTok{, }\DataTypeTok{country=}\OtherTok{NULL}\NormalTok{) \{}
  \NormalTok{if(!}\KeywordTok{is.null}\NormalTok{(year)) \{}
    \NormalTok{dat <-}\StringTok{ }\NormalTok{dat[dat$year %in%}\StringTok{ }\NormalTok{year, ]}
  \NormalTok{\}}
  \NormalTok{if (!}\KeywordTok{is.null}\NormalTok{(country)) \{}
    \NormalTok{dat <-}\StringTok{ }\NormalTok{dat[dat$country %in%}\StringTok{ }\NormalTok{country,]}
  \NormalTok{\}}
  \NormalTok{gdp <-}\StringTok{ }\NormalTok{dat$pop *}\StringTok{ }\NormalTok{dat$gdpPercap}
  
  \NormalTok{new <-}\StringTok{ }\KeywordTok{cbind}\NormalTok{(dat, }\DataTypeTok{gdp=}\NormalTok{gdp)}
  \KeywordTok{return}\NormalTok{(new)}
\NormalTok{\}}
\NormalTok{plyr::}\KeywordTok{ddply}\NormalTok{(}
  \DataTypeTok{.data =} \KeywordTok{calcGDP}\NormalTok{(gapminderFiveYearData),}
  \DataTypeTok{.variables =} \StringTok{"continent"}\NormalTok{,}
  \DataTypeTok{.fun =} \NormalTok{function(x) }\KeywordTok{mean}\NormalTok{(x$gdp)}
\NormalTok{)}
\end{Highlighting}
\end{Shaded}

\begin{verbatim}
##   continent           V1
## 1    Africa  20904782844
## 2  Americas 379262350210
## 3      Asia 227233738153
## 4    Europe 269442085301
## 5   Oceania 188187105354
\end{verbatim}

Yeah you get the idea.

Data tables have built-in ``methods'' for a range of functions, these
are often faster than standard dataframes or matrices, if these aren't
found it uses dataframe functions. A ``Data Table'' is compatible with
any function from any package designed for a ``Data Frame''.

\subsection{File I/O (Input/Output)}\label{file-io-inputoutput}

fread is ``fast read'', and it's \textbf{fast}, even for large data
files. Let's try it out on some larger datafiles:

\begin{Shaded}
\begin{Highlighting}[]
\NormalTok{gapminderlarge <-}\StringTok{ }\KeywordTok{fread}\NormalTok{(}\StringTok{"gapminder-large.csv"}\NormalTok{, }\DataTypeTok{header=}\NormalTok{T)}
\end{Highlighting}
\end{Shaded}

fread is smart, it auto detects column classes, separators, headers,
nrows (for a regularly separated file). We can use the same comand for a
whole bunch of file formats. All the usual reading options can be
specified manually\ldots{}

\begin{Shaded}
\begin{Highlighting}[]
\NormalTok{gapminderFiveYearData <-}\StringTok{ }\KeywordTok{fread}\NormalTok{(}\StringTok{"gapminder-FiveYearData.tsv"}\NormalTok{) }\CommentTok{#tab delimited}
\NormalTok{gapminderFiveYearData <-}\StringTok{ }\KeywordTok{fread}\NormalTok{(}\StringTok{"gapminder-FiveYearData.txt"}\NormalTok{) }\CommentTok{#space delimited}
\NormalTok{gapminderFiveYearDataCrop <-}\StringTok{ }\KeywordTok{fread}\NormalTok{(}\StringTok{"gapminder-FiveYearData.tsv"}\NormalTok{, }\DataTypeTok{header=}\NormalTok{T, }\DataTypeTok{col.names=}\KeywordTok{c}\NormalTok{(}\StringTok{"place"}\NormalTok{, }\StringTok{"time"}\NormalTok{, }\StringTok{"people"}\NormalTok{, }\StringTok{"big place"}\NormalTok{, }\StringTok{"life"}\NormalTok{, }\StringTok{"money"}\NormalTok{), }\DataTypeTok{nrows=}\DecValTok{1000}\NormalTok{, }\DataTypeTok{stringsAsFactors=}\NormalTok{F)}
\NormalTok{gapminderFiveYearDataCrop}
\end{Highlighting}
\end{Shaded}

\begin{verbatim}
##             place time    people big place   life      money
##    1: Afghanistan 1952   8425333      Asia 28.801   779.4453
##    2: Afghanistan 1957   9240934      Asia 30.332   820.8530
##    3: Afghanistan 1962  10267083      Asia 31.997   853.1007
##    4: Afghanistan 1967  11537966      Asia 34.020   836.1971
##    5: Afghanistan 1972  13079460      Asia 36.088   739.9811
##   ---                                                       
##  996:      Mexico 2007 108700891  Americas 76.195 11977.5750
##  997:    Mongolia 1952    800663      Asia 42.244   786.5669
##  998:    Mongolia 1957    882134      Asia 45.248   912.6626
##  999:    Mongolia 1962   1010280      Asia 48.251  1056.3540
## 1000:    Mongolia 1967   1149500      Asia 51.253  1226.0411
\end{verbatim}

\ldots{}but it does a lot of the tedious work for you (pretty well too).

It's also got cool progress bars for large files :) These kick in
automatically if the file takes longer than about a second. This is
really handy to know your code is working, and how long it will take.

\begin{Shaded}
\begin{Highlighting}[]
\NormalTok{gapminderlarger <-}\StringTok{ }\KeywordTok{fread}\NormalTok{(}\StringTok{"gapminder-larger.csv"}\NormalTok{)}
\end{Highlighting}
\end{Shaded}

\begin{verbatim}
## 
Read 42.7% of 6625152 rows
Read 71.2% of 6625152 rows
Read 99.5% of 6625152 rows
Read 6625152 rows and 6 (of 6) columns from 0.321 GB file in 00:00:05
\end{verbatim}

It's so fast it tells you. Let's compare that with base R:

\begin{Shaded}
\begin{Highlighting}[]
\KeywordTok{system.time}\NormalTok{(gapminderlarger.dataframe <-}\StringTok{ }\KeywordTok{read.csv}\NormalTok{(}\StringTok{"gapminder-larger.csv"}\NormalTok{, }\DataTypeTok{header=}\NormalTok{T))}
\end{Highlighting}
\end{Shaded}

\begin{verbatim}
##    user  system elapsed 
##  22.128   0.288  22.440
\end{verbatim}

The same operation took much longer with base R, with larger files (or
repeating this many times) that \textasciitilde{}6x difference could
mean a lot for your workflow.

FYI - there's also a ``fast write'' compatible with several file formats

\begin{Shaded}
\begin{Highlighting}[]
\KeywordTok{fwrite}\NormalTok{(gapminderlarger, }\DataTypeTok{file=}\StringTok{"test.csv"}\NormalTok{) }\CommentTok{#defaults to csv}
\KeywordTok{fwrite}\NormalTok{(gapminderlarger, }\DataTypeTok{file=}\StringTok{"test.tsv"}\NormalTok{, }\DataTypeTok{sep=}\StringTok{"}\CharTok{\textbackslash{}t}\StringTok{"}\NormalTok{)}
\end{Highlighting}
\end{Shaded}

They're also fast to write data, compared to base R:

\begin{Shaded}
\begin{Highlighting}[]
\KeywordTok{system.time}\NormalTok{(}\KeywordTok{fwrite}\NormalTok{(gapminderlarger, }\DataTypeTok{file=}\StringTok{"test.csv"}\NormalTok{))}
\end{Highlighting}
\end{Shaded}

\begin{verbatim}
##    user  system elapsed 
##  17.616   0.440  19.775
\end{verbatim}

\begin{Shaded}
\begin{Highlighting}[]
\KeywordTok{system.time}\NormalTok{(}\KeywordTok{write.csv}\NormalTok{(gapminderlarger, }\DataTypeTok{file=}\StringTok{"test.csv"}\NormalTok{))}
\end{Highlighting}
\end{Shaded}

\begin{verbatim}
##    user  system elapsed 
##  39.484   0.396  42.705
\end{verbatim}

\subsection{readr (Hadley Wickham and
RStudio)}\label{readr-hadley-wickham-and-rstudio}

Another package enables faster alternatives to existing read functions
in base R: these work almost exactly the same as their base R
counterparts.

\begin{longtable}[c]{@{}lll@{}}
\toprule
\textbf{base R} & \textbf{readr}\tabularnewline
\midrule
\endhead
spaced file & \texttt{read.table} & \texttt{read\_table}\tabularnewline
fixed-width file & \texttt{read.fwf} & \texttt{read\_fwf}\tabularnewline
comma-separated file & \texttt{read.csv} &
\texttt{read\_csv}\tabularnewline
semicolon-separated file & \texttt{read.csv2} &
\texttt{read\_csv2}\tabularnewline
tab-delimited file & \texttt{read.table} &
\texttt{read\_tsv}\tabularnewline
comma-separated file & \texttt{read.csv} &
\texttt{read\_csv}\tabularnewline
file or string \texttt{readLines} & \texttt{read\_lines} or
\texttt{read\_file}\tabularnewline
\bottomrule
\end{longtable}

Let's try it out on a space-delimited file:

\begin{Shaded}
\begin{Highlighting}[]
\KeywordTok{library}\NormalTok{(}\StringTok{"readr"}\NormalTok{)}
\KeywordTok{system.time}\NormalTok{(}\KeywordTok{read_table}\NormalTok{(}\StringTok{"gapminder-FiveYearData.txt"}\NormalTok{))}
\end{Highlighting}
\end{Shaded}

\begin{verbatim}
##    user  system elapsed 
##   0.024   0.000   0.022
\end{verbatim}

\begin{Shaded}
\begin{Highlighting}[]
\KeywordTok{system.time}\NormalTok{(}\KeywordTok{read.table}\NormalTok{(}\StringTok{"gapminder-FiveYearData.txt"}\NormalTok{))}
\end{Highlighting}
\end{Shaded}

\begin{verbatim}
##    user  system elapsed 
##   0.012   0.000   0.012
\end{verbatim}

Even on a small file \texttt{readr} is faster than base R. This also
holds for larger csv files:

\begin{Shaded}
\begin{Highlighting}[]
\KeywordTok{system.time}\NormalTok{(}\KeywordTok{read_csv}\NormalTok{(}\StringTok{"gapminder-larger.csv"}\NormalTok{))}
\end{Highlighting}
\end{Shaded}

\begin{verbatim}
##    user  system elapsed 
##   4.452   0.040   4.499
\end{verbatim}

\begin{Shaded}
\begin{Highlighting}[]
\KeywordTok{system.time}\NormalTok{(}\KeywordTok{read.csv}\NormalTok{(}\StringTok{"gapminder-larger.csv"}\NormalTok{))}
\end{Highlighting}
\end{Shaded}

\begin{verbatim}
##    user  system elapsed 
##  21.960   0.348  22.333
\end{verbatim}

\texttt{readr} also has a handy progress bar allowign us to monitor
progress. There is an equivalent \texttt{readxl} package with a
\texttt{read\_excel} function compatible with xls or xlsx files and
enables sheet selection. This is a relatively new alternative to the
\texttt{xlsx} package and it's \texttt{read.xlsx} function which are
difficult to work with (as it is java and perl dependent).

\subsection{Another solution:
bigmemory}\label{another-solution-bigmemory}

\begin{Shaded}
\begin{Highlighting}[]
\KeywordTok{library}\NormalTok{(}\StringTok{"bigmemory"}\NormalTok{)}
\end{Highlighting}
\end{Shaded}

``bigmemory'' uses the ``big.matrix'' format to access large data files
in a C++ framework - rather than stored in RAM/memory as usual in R.
This is handy for handling \textbf{very large} files, when loading the
full dataset in working environment (RAM memory) slows your computer to
a halt. Might be handy on servers / HPC too but usually they have enough
memory if you're willing to wait for it in a queue.

Let's try out bigmemory, first we convert an R data matrix into a
``big.matrix'':

\begin{Shaded}
\begin{Highlighting}[]
\NormalTok{gapminderFiveYearData.big <-}\StringTok{ }\KeywordTok{as.big.matrix}\NormalTok{(gapminderFiveYearData)}
\NormalTok{gapminderFiveYearData.big}
\end{Highlighting}
\end{Shaded}

\begin{verbatim}
## An object of class "big.matrix"
## Slot "address":
## <pointer: 0x9455710>
\end{verbatim}

\begin{Shaded}
\begin{Highlighting}[]
\KeywordTok{class}\NormalTok{(gapminderFiveYearData.big)}
\end{Highlighting}
\end{Shaded}

\begin{verbatim}
## [1] "big.matrix"
## attr(,"package")
## [1] "bigmemory"
\end{verbatim}

\begin{Shaded}
\begin{Highlighting}[]
\KeywordTok{dim}\NormalTok{(gapminderFiveYearData.big)}
\end{Highlighting}
\end{Shaded}

\begin{verbatim}
## [1] 1704    6
\end{verbatim}

\begin{Shaded}
\begin{Highlighting}[]
\KeywordTok{head}\NormalTok{(gapminderFiveYearData.big)}
\end{Highlighting}
\end{Shaded}

\begin{verbatim}
##   country year      pop continent lifeExp gdpPercap
## 1       1 1952  8425333         3  28.801  779.4453
## 2       1 1957  9240934         3  30.332  820.8530
## 3       1 1962 10267083         3  31.997  853.1007
## 4       1 1967 11537966         3  34.020  836.1971
## 5       1 1972 13079460         3  36.088  739.9811
## 6       1 1977 14880372         3  38.438  786.1134
\end{verbatim}

\begin{Shaded}
\begin{Highlighting}[]
\KeywordTok{tail}\NormalTok{(gapminderFiveYearData.big)}
\end{Highlighting}
\end{Shaded}

\begin{verbatim}
##      country year      pop continent lifeExp gdpPercap
## 1699     142 1982  7636524         1  60.363  788.8550
## 1700     142 1987  9216418         1  62.351  706.1573
## 1701     142 1992 10704340         1  60.377  693.4208
## 1702     142 1997 11404948         1  46.809  792.4500
## 1703     142 2002 11926563         1  39.989  672.0386
## 1704     142 2007 12311143         1  43.487  469.7093
\end{verbatim}

\begin{Shaded}
\begin{Highlighting}[]
\KeywordTok{str}\NormalTok{(gapminderFiveYearData.big)}
\end{Highlighting}
\end{Shaded}

\begin{verbatim}
## Formal class 'big.matrix' [package "bigmemory"] with 1 slot
##   ..@ address:<externalptr>
\end{verbatim}

bigmemory, also has read/write functions direct to big.matrix format:

\begin{Shaded}
\begin{Highlighting}[]
\KeywordTok{write.big.matrix}\NormalTok{(gapminderFiveYearData.big, }\StringTok{"test.csv"}\NormalTok{)}
\NormalTok{gapminderFiveYearData.big <-}\StringTok{ }\KeywordTok{read.big.matrix}\NormalTok{(}\StringTok{"test.csv"}\NormalTok{)}
\end{Highlighting}
\end{Shaded}

These are designed to be efficient for memory - how fast are they?

\begin{Shaded}
\begin{Highlighting}[]
\KeywordTok{system.time}\NormalTok{(gapminderlarger.big <-}\StringTok{ }\KeywordTok{read.big.matrix}\NormalTok{(}\StringTok{"gapminder-larger.csv"}\NormalTok{))}
\end{Highlighting}
\end{Shaded}

\begin{verbatim}
##    user  system elapsed 
##  12.680   0.188  12.883
\end{verbatim}

\begin{Shaded}
\begin{Highlighting}[]
\KeywordTok{system.time}\NormalTok{(}\KeywordTok{write.big.matrix}\NormalTok{(gapminderFiveYearData.big, }\StringTok{"test.csv"}\NormalTok{))}
\end{Highlighting}
\end{Shaded}

\begin{verbatim}
##    user  system elapsed 
##   0.012   0.000   0.012
\end{verbatim}

\subsection{New and Shiny: FEATHER}\label{new-and-shiny-feather}

\subsubsection{A Fast On-Disk Format for Data Frames for R and Python,
powered by Apache
Arrow}\label{a-fast-on-disk-format-for-data-frames-for-r-and-python-powered-by-apache-arrow}

FEATHER (is it's own fast file format) - from Hadley Wickham
ggplot/dplyr/etc\ldots{} and Wes Mckinney (pandas in Python) Note: it's
in development (unstable) - future versions may not read past versions -
intended for use to transfer files quickly (e.g., between R and Python)

At the moment you can only try it out from their github repo (in R or
python), it will no doubt end up on CRAN very soon:

\begin{Shaded}
\begin{Highlighting}[]
\KeywordTok{library}\NormalTok{(}\StringTok{"devtools"}\NormalTok{)}
\NormalTok{devtools::}\KeywordTok{install_github}\NormalTok{(}\StringTok{"wesm/feather/R"}\NormalTok{)}
\KeywordTok{library}\NormalTok{(feather)}
\end{Highlighting}
\end{Shaded}

FEATHER has it's own file I/O commands (and format):

\begin{Shaded}
\begin{Highlighting}[]
\NormalTok{path <-}\StringTok{ "gapminder-FiveYearData.feather"}
\KeywordTok{write_feather}\NormalTok{(gapminderFiveYearData, path) }\CommentTok{#write data frame to file}
\NormalTok{gapminderFiveYearData <-}\StringTok{ }\KeywordTok{read_feather}\NormalTok{(path) }\CommentTok{#read to data frame}
\NormalTok{gapminderFiveYearData}
\end{Highlighting}
\end{Shaded}

\begin{verbatim}
## Source: local data frame [1,704 x 6]
## 
##        country  year      pop continent lifeExp gdpPercap
##          <chr> <int>    <dbl>     <chr>   <dbl>     <dbl>
## 1  Afghanistan  1952  8425333      Asia  28.801  779.4453
## 2  Afghanistan  1957  9240934      Asia  30.332  820.8530
## 3  Afghanistan  1962 10267083      Asia  31.997  853.1007
## 4  Afghanistan  1967 11537966      Asia  34.020  836.1971
## 5  Afghanistan  1972 13079460      Asia  36.088  739.9811
## 6  Afghanistan  1977 14880372      Asia  38.438  786.1134
## 7  Afghanistan  1982 12881816      Asia  39.854  978.0114
## 8  Afghanistan  1987 13867957      Asia  40.822  852.3959
## 9  Afghanistan  1992 16317921      Asia  41.674  649.3414
## 10 Afghanistan  1997 22227415      Asia  41.763  635.3414
## ..         ...   ...      ...       ...     ...       ...
\end{verbatim}

Did I mention it's crazy fast?

\begin{Shaded}
\begin{Highlighting}[]
\NormalTok{path <-}\StringTok{ "gapminderlarger.feather"}
\KeywordTok{system.time}\NormalTok{(}\KeywordTok{write_feather}\NormalTok{(gapminderlarger, path))}
\end{Highlighting}
\end{Shaded}

\begin{verbatim}
##    user  system elapsed 
##   0.304   0.224   1.570
\end{verbatim}

\begin{Shaded}
\begin{Highlighting}[]
\KeywordTok{system.time}\NormalTok{(gapminderlarger.feather <-}\StringTok{ }\KeywordTok{read_feather}\NormalTok{(path))}
\end{Highlighting}
\end{Shaded}

\begin{verbatim}
##    user  system elapsed 
##   0.332   0.024   0.358
\end{verbatim}

Or install and run in Python:

\begin{verbatim}
import feather
path = 'my_data.feather'
feather.write_dataframe(df, path)
df = feather.read_dataframe(path)
\end{verbatim}

Note that FEATHER is designed for data \emph{already} loaded into python
or R.

\subsection{FILE I/O Summary}\label{file-io-summary}

\subsubsection{READ}\label{read}

\begin{longtable}[c]{@{}lllll@{}}
\toprule
\begin{minipage}[b]{0.05\columnwidth}\raggedright\strut
\textbf{base R}
\strut\end{minipage} &
\begin{minipage}[b]{0.05\columnwidth}\raggedright\strut
\textbf{data table}
\strut\end{minipage} &
\begin{minipage}[b]{0.05\columnwidth}\raggedright\strut
\textbf{readr}
\strut\end{minipage} &
\begin{minipage}[b]{0.05\columnwidth}\raggedright\strut
\textbf{bigmemory}
\strut\end{minipage} &
\begin{minipage}[b]{0.05\columnwidth}\raggedright\strut
\textbf{feather}
\strut\end{minipage}\tabularnewline
\midrule
\endhead
\begin{minipage}[t]{0.05\columnwidth}\raggedright\strut
\texttt{read.csv}
\strut\end{minipage} &
\begin{minipage}[t]{0.05\columnwidth}\raggedright\strut
\texttt{fread}
\strut\end{minipage} &
\begin{minipage}[t]{0.05\columnwidth}\raggedright\strut
\texttt{read\_csv}
\strut\end{minipage} &
\begin{minipage}[t]{0.05\columnwidth}\raggedright\strut
\texttt{read.big.matrix}
\strut\end{minipage} &
\begin{minipage}[t]{0.05\columnwidth}\raggedright\strut
\texttt{read\_feather}
\strut\end{minipage}\tabularnewline
\begin{minipage}[t]{0.05\columnwidth}\raggedright\strut
52.203s
\strut\end{minipage} &
\begin{minipage}[t]{0.05\columnwidth}\raggedright\strut
8.154s
\strut\end{minipage} &
\begin{minipage}[t]{0.05\columnwidth}\raggedright\strut
11.120s
\strut\end{minipage} &
\begin{minipage}[t]{0.05\columnwidth}\raggedright\strut
28.647s
\strut\end{minipage} &
\begin{minipage}[t]{0.05\columnwidth}\raggedright\strut
2.414s
\strut\end{minipage}\tabularnewline
\bottomrule
\end{longtable}

\subsubsection{Convert dataframe to
format}\label{convert-dataframe-to-format}

\begin{longtable}[c]{@{}llll@{}}
\toprule
\textbf{base R} & \textbf{data table} & \textbf{bigmemory} &
\textbf{feather}\tabularnewline
\midrule
\endhead
\texttt{data.frame} & \texttt{as.data.table} & \texttt{as.big.matrix} &
built-in\tabularnewline
NA & 0.002s & 66.07s & NA\tabularnewline
\bottomrule
\end{longtable}

\subsubsection{Write}\label{write}

\begin{longtable}[c]{@{}llll@{}}
\toprule
\textbf{base R} & \textbf{data table} & \textbf{bigmemory} &
\textbf{feather}\tabularnewline
\midrule
\endhead
\texttt{write.csv} & \texttt{fwrite} & \texttt{write.big.matrix} &
\texttt{write\_feather}\tabularnewline
71.382s & 35.453s & 0.068ss & 5.008s\tabularnewline
\bottomrule
\end{longtable}

\subsection{Manipulating Data Tables}\label{manipulating-data-tables}

\begin{Shaded}
\begin{Highlighting}[]
\NormalTok{gapminderFiveYearData <-}\StringTok{ }\KeywordTok{fread}\NormalTok{(}\StringTok{"gapminder-FiveYearData.csv"}\NormalTok{, }\DataTypeTok{data.table=}\NormalTok{T, }\DataTypeTok{header =} \NormalTok{T)}
\KeywordTok{class}\NormalTok{(gapminderFiveYearData)}
\end{Highlighting}
\end{Shaded}

\begin{verbatim}
## [1] "data.table" "data.frame"
\end{verbatim}

We can simply treat it as a data frame in many cases:

\begin{Shaded}
\begin{Highlighting}[]
\NormalTok{gapminderFiveYearData[}\DecValTok{1}\NormalTok{,]}
\end{Highlighting}
\end{Shaded}

\begin{verbatim}
##        country year     pop continent lifeExp gdpPercap
## 1: Afghanistan 1952 8425333      Asia  28.801  779.4453
\end{verbatim}

\begin{Shaded}
\begin{Highlighting}[]
\KeywordTok{colnames}\NormalTok{(gapminderFiveYearData)}
\end{Highlighting}
\end{Shaded}

\begin{verbatim}
## [1] "country"   "year"      "pop"       "continent" "lifeExp"   "gdpPercap"
\end{verbatim}

\begin{Shaded}
\begin{Highlighting}[]
\KeywordTok{head}\NormalTok{(gapminderFiveYearData$country)}
\end{Highlighting}
\end{Shaded}

\begin{verbatim}
## [1] "Afghanistan" "Afghanistan" "Afghanistan" "Afghanistan" "Afghanistan"
## [6] "Afghanistan"
\end{verbatim}

\begin{Shaded}
\begin{Highlighting}[]
\KeywordTok{tail}\NormalTok{(gapminderFiveYearData$country)}
\end{Highlighting}
\end{Shaded}

\begin{verbatim}
## [1] "Zimbabwe" "Zimbabwe" "Zimbabwe" "Zimbabwe" "Zimbabwe" "Zimbabwe"
\end{verbatim}

Data Table has a ``Natural'' Syntax

\texttt{DT{[}where,\ select\textbar{}update\textbar{}do,\ by{]}}

\ldots{}although suspiciously similar to SQL?

it allows chaining queries: \texttt{DT{[}{]}{[}{]}}

Formally: we subset a datatable, Dt, with \texttt{DT{[}i,\ j,\ by{]}}

\subsubsection{I: row selection}\label{i-row-selection}

\begin{Shaded}
\begin{Highlighting}[]
\NormalTok{gapminderFiveYearData[}\KeywordTok{c}\NormalTok{(}\DecValTok{1}\NormalTok{:}\DecValTok{5}\NormalTok{, }\DecValTok{100}\NormalTok{:}\DecValTok{105}\NormalTok{),] }\CommentTok{#by number}
\end{Highlighting}
\end{Shaded}

\begin{verbatim}
##         country year       pop continent lifeExp gdpPercap
##  1: Afghanistan 1952   8425333      Asia  28.801  779.4453
##  2: Afghanistan 1957   9240934      Asia  30.332  820.8530
##  3: Afghanistan 1962  10267083      Asia  31.997  853.1007
##  4: Afghanistan 1967  11537966      Asia  34.020  836.1971
##  5: Afghanistan 1972  13079460      Asia  36.088  739.9811
##  6:  Bangladesh 1967  62821884      Asia  43.453  721.1861
##  7:  Bangladesh 1972  70759295      Asia  45.252  630.2336
##  8:  Bangladesh 1977  80428306      Asia  46.923  659.8772
##  9:  Bangladesh 1982  93074406      Asia  50.009  676.9819
## 10:  Bangladesh 1987 103764241      Asia  52.819  751.9794
## 11:  Bangladesh 1992 113704579      Asia  56.018  837.8102
\end{verbatim}

\begin{Shaded}
\begin{Highlighting}[]
\NormalTok{gapminderFiveYearData[gapminderFiveYearData$country==}\StringTok{"New Zealand"}\NormalTok{,] }\CommentTok{#by condition}
\end{Highlighting}
\end{Shaded}

\begin{verbatim}
##         country year     pop continent lifeExp gdpPercap
##  1: New Zealand 1952 1994794   Oceania  69.390  10556.58
##  2: New Zealand 1957 2229407   Oceania  70.260  12247.40
##  3: New Zealand 1962 2488550   Oceania  71.240  13175.68
##  4: New Zealand 1967 2728150   Oceania  71.520  14463.92
##  5: New Zealand 1972 2929100   Oceania  71.890  16046.04
##  6: New Zealand 1977 3164900   Oceania  72.220  16233.72
##  7: New Zealand 1982 3210650   Oceania  73.840  17632.41
##  8: New Zealand 1987 3317166   Oceania  74.320  19007.19
##  9: New Zealand 1992 3437674   Oceania  76.330  18363.32
## 10: New Zealand 1997 3676187   Oceania  77.550  21050.41
## 11: New Zealand 2002 3908037   Oceania  79.110  23189.80
## 12: New Zealand 2007 4115771   Oceania  80.204  25185.01
\end{verbatim}

\begin{Shaded}
\begin{Highlighting}[]
\NormalTok{gapminderFiveYearData[gapminderFiveYearData$country %in%}\StringTok{ }\KeywordTok{c}\NormalTok{(}\StringTok{"New Zealand"}\NormalTok{, }\StringTok{"Australia"}\NormalTok{, }\StringTok{"Japan"}\NormalTok{),] }\CommentTok{#by condition}
\end{Highlighting}
\end{Shaded}

\begin{verbatim}
##         country year       pop continent lifeExp gdpPercap
##  1:   Australia 1952   8691212   Oceania  69.120 10039.596
##  2:   Australia 1957   9712569   Oceania  70.330 10949.650
##  3:   Australia 1962  10794968   Oceania  70.930 12217.227
##  4:   Australia 1967  11872264   Oceania  71.100 14526.125
##  5:   Australia 1972  13177000   Oceania  71.930 16788.629
##  6:   Australia 1977  14074100   Oceania  73.490 18334.198
##  7:   Australia 1982  15184200   Oceania  74.740 19477.009
##  8:   Australia 1987  16257249   Oceania  76.320 21888.889
##  9:   Australia 1992  17481977   Oceania  77.560 23424.767
## 10:   Australia 1997  18565243   Oceania  78.830 26997.937
## 11:   Australia 2002  19546792   Oceania  80.370 30687.755
## 12:   Australia 2007  20434176   Oceania  81.235 34435.367
## 13:       Japan 1952  86459025      Asia  63.030  3216.956
## 14:       Japan 1957  91563009      Asia  65.500  4317.694
## 15:       Japan 1962  95831757      Asia  68.730  6576.649
## 16:       Japan 1967 100825279      Asia  71.430  9847.789
## 17:       Japan 1972 107188273      Asia  73.420 14778.786
## 18:       Japan 1977 113872473      Asia  75.380 16610.377
## 19:       Japan 1982 118454974      Asia  77.110 19384.106
## 20:       Japan 1987 122091325      Asia  78.670 22375.942
## 21:       Japan 1992 124329269      Asia  79.360 26824.895
## 22:       Japan 1997 125956499      Asia  80.690 28816.585
## 23:       Japan 2002 127065841      Asia  82.000 28604.592
## 24:       Japan 2007 127467972      Asia  82.603 31656.068
## 25: New Zealand 1952   1994794   Oceania  69.390 10556.576
## 26: New Zealand 1957   2229407   Oceania  70.260 12247.395
## 27: New Zealand 1962   2488550   Oceania  71.240 13175.678
## 28: New Zealand 1967   2728150   Oceania  71.520 14463.919
## 29: New Zealand 1972   2929100   Oceania  71.890 16046.037
## 30: New Zealand 1977   3164900   Oceania  72.220 16233.718
## 31: New Zealand 1982   3210650   Oceania  73.840 17632.410
## 32: New Zealand 1987   3317166   Oceania  74.320 19007.191
## 33: New Zealand 1992   3437674   Oceania  76.330 18363.325
## 34: New Zealand 1997   3676187   Oceania  77.550 21050.414
## 35: New Zealand 2002   3908037   Oceania  79.110 23189.801
## 36: New Zealand 2007   4115771   Oceania  80.204 25185.009
##         country year       pop continent lifeExp gdpPercap
\end{verbatim}

\begin{Shaded}
\begin{Highlighting}[]
\NormalTok{gapminderFiveYearData[year==}\StringTok{"1952"}\NormalTok{]}
\end{Highlighting}
\end{Shaded}

\begin{verbatim}
##                 country year      pop continent lifeExp gdpPercap
##   1:        Afghanistan 1952  8425333      Asia  28.801  779.4453
##   2:            Albania 1952  1282697    Europe  55.230 1601.0561
##   3:            Algeria 1952  9279525    Africa  43.077 2449.0082
##   4:             Angola 1952  4232095    Africa  30.015 3520.6103
##   5:          Argentina 1952 17876956  Americas  62.485 5911.3151
##  ---                                                             
## 138:            Vietnam 1952 26246839      Asia  40.412  605.0665
## 139: West Bank and Gaza 1952  1030585      Asia  43.160 1515.5923
## 140:         Yemen Rep. 1952  4963829      Asia  32.548  781.7176
## 141:             Zambia 1952  2672000    Africa  42.038 1147.3888
## 142:           Zimbabwe 1952  3080907    Africa  48.451  406.8841
\end{verbatim}

\begin{Shaded}
\begin{Highlighting}[]
\KeywordTok{setkey}\NormalTok{(gapminderFiveYearData, country)}
\NormalTok{gapminderFiveYearData[}\KeywordTok{c}\NormalTok{(}\StringTok{"New Zealand"}\NormalTok{,}\StringTok{"Australia"}\NormalTok{)] }\CommentTok{#by key (will be detailed later)}
\end{Highlighting}
\end{Shaded}

\begin{verbatim}
##         country year      pop continent lifeExp gdpPercap
##  1: New Zealand 1952  1994794   Oceania  69.390  10556.58
##  2: New Zealand 1957  2229407   Oceania  70.260  12247.40
##  3: New Zealand 1962  2488550   Oceania  71.240  13175.68
##  4: New Zealand 1967  2728150   Oceania  71.520  14463.92
##  5: New Zealand 1972  2929100   Oceania  71.890  16046.04
##  6: New Zealand 1977  3164900   Oceania  72.220  16233.72
##  7: New Zealand 1982  3210650   Oceania  73.840  17632.41
##  8: New Zealand 1987  3317166   Oceania  74.320  19007.19
##  9: New Zealand 1992  3437674   Oceania  76.330  18363.32
## 10: New Zealand 1997  3676187   Oceania  77.550  21050.41
## 11: New Zealand 2002  3908037   Oceania  79.110  23189.80
## 12: New Zealand 2007  4115771   Oceania  80.204  25185.01
## 13:   Australia 1952  8691212   Oceania  69.120  10039.60
## 14:   Australia 1957  9712569   Oceania  70.330  10949.65
## 15:   Australia 1962 10794968   Oceania  70.930  12217.23
## 16:   Australia 1967 11872264   Oceania  71.100  14526.12
## 17:   Australia 1972 13177000   Oceania  71.930  16788.63
## 18:   Australia 1977 14074100   Oceania  73.490  18334.20
## 19:   Australia 1982 15184200   Oceania  74.740  19477.01
## 20:   Australia 1987 16257249   Oceania  76.320  21888.89
## 21:   Australia 1992 17481977   Oceania  77.560  23424.77
## 22:   Australia 1997 18565243   Oceania  78.830  26997.94
## 23:   Australia 2002 19546792   Oceania  80.370  30687.75
## 24:   Australia 2007 20434176   Oceania  81.235  34435.37
##         country year      pop continent lifeExp gdpPercap
\end{verbatim}

\subsubsection{J: column selection}\label{j-column-selection}

\begin{Shaded}
\begin{Highlighting}[]
\KeywordTok{head}\NormalTok{(gapminderFiveYearData[,}\StringTok{"country"}\NormalTok{]) }\CommentTok{#by names}
\end{Highlighting}
\end{Shaded}

\begin{verbatim}
## [1] "country"
\end{verbatim}

\begin{Shaded}
\begin{Highlighting}[]
\KeywordTok{head}\NormalTok{(gapminderFiveYearData[,country]) }\CommentTok{#by column}
\end{Highlighting}
\end{Shaded}

\begin{verbatim}
## [1] "Afghanistan" "Afghanistan" "Afghanistan" "Afghanistan" "Afghanistan"
## [6] "Afghanistan"
\end{verbatim}

\begin{Shaded}
\begin{Highlighting}[]
\NormalTok{gapminderFiveYearData[,}\KeywordTok{list}\NormalTok{(country, year, pop)] }\CommentTok{#by list}
\end{Highlighting}
\end{Shaded}

\begin{verbatim}
##           country year      pop
##    1: Afghanistan 1952  8425333
##    2: Afghanistan 1957  9240934
##    3: Afghanistan 1962 10267083
##    4: Afghanistan 1967 11537966
##    5: Afghanistan 1972 13079460
##   ---                          
## 1700:    Zimbabwe 1987  9216418
## 1701:    Zimbabwe 1992 10704340
## 1702:    Zimbabwe 1997 11404948
## 1703:    Zimbabwe 2002 11926563
## 1704:    Zimbabwe 2007 12311143
\end{verbatim}

This allows operations to be performed on columns:

\begin{Shaded}
\begin{Highlighting}[]
\NormalTok{gapminderFiveYearData[,}\KeywordTok{sum}\NormalTok{(gdpPercap)] }\CommentTok{#by colnames}
\end{Highlighting}
\end{Shaded}

\begin{verbatim}
## [1] 12294917
\end{verbatim}

\begin{Shaded}
\begin{Highlighting}[]
\NormalTok{gapminderFiveYearData[,}\KeywordTok{sum}\NormalTok{(gdpPercap*pop)] }\CommentTok{#by colnames}
\end{Highlighting}
\end{Shaded}

\begin{verbatim}
## [1] 3.183235e+14
\end{verbatim}

\begin{Shaded}
\begin{Highlighting}[]
\NormalTok{gapminderFiveYearData[,}\KeywordTok{mean}\NormalTok{(pop)] }\CommentTok{#by colnames}
\end{Highlighting}
\end{Shaded}

\begin{verbatim}
## [1] 29601212
\end{verbatim}

\begin{Shaded}
\begin{Highlighting}[]
\NormalTok{gapminderFiveYearData[,}\KeywordTok{mean}\NormalTok{(lifeExp)] }\CommentTok{#by colnames}
\end{Highlighting}
\end{Shaded}

\begin{verbatim}
## [1] 59.47444
\end{verbatim}

\subsubsection{BY: group operation}\label{by-group-operation}

This is paricularly power in that we can apply operations to sets
values, grouped ``by'':

\begin{Shaded}
\begin{Highlighting}[]
\NormalTok{gapminderFiveYearData[j=}\KeywordTok{sum}\NormalTok{(gdpPercap), by=year]}
\end{Highlighting}
\end{Shaded}

\begin{verbatim}
##     year        V1
##  1: 1952  528989.2
##  2: 1957  610516.0
##  3: 1962  671065.4
##  4: 1967  778678.7
##  5: 1972  961351.8
##  6: 1977 1038469.6
##  7: 1982 1067684.0
##  8: 1987 1121930.7
##  9: 1992 1158522.4
## 10: 1997 1290804.9
## 11: 2002 1408334.5
## 12: 2007 1658570.2
\end{verbatim}

\begin{Shaded}
\begin{Highlighting}[]
\NormalTok{gapminderFiveYearData[,}\KeywordTok{sum}\NormalTok{(gdpPercap), year]}
\end{Highlighting}
\end{Shaded}

\begin{verbatim}
##     year        V1
##  1: 1952  528989.2
##  2: 1957  610516.0
##  3: 1962  671065.4
##  4: 1967  778678.7
##  5: 1972  961351.8
##  6: 1977 1038469.6
##  7: 1982 1067684.0
##  8: 1987 1121930.7
##  9: 1992 1158522.4
## 10: 1997 1290804.9
## 11: 2002 1408334.5
## 12: 2007 1658570.2
\end{verbatim}

\begin{Shaded}
\begin{Highlighting}[]
\NormalTok{gapminderFiveYearData[,}\KeywordTok{mean}\NormalTok{(lifeExp), year]}
\end{Highlighting}
\end{Shaded}

\begin{verbatim}
##     year       V1
##  1: 1952 49.05762
##  2: 1957 51.50740
##  3: 1962 53.60925
##  4: 1967 55.67829
##  5: 1972 57.64739
##  6: 1977 59.57016
##  7: 1982 61.53320
##  8: 1987 63.21261
##  9: 1992 64.16034
## 10: 1997 65.01468
## 11: 2002 65.69492
## 12: 2007 67.00742
\end{verbatim}

\begin{Shaded}
\begin{Highlighting}[]
\NormalTok{gapminderFiveYearData[,}\KeywordTok{sum}\NormalTok{(pop), by=}\KeywordTok{list}\NormalTok{(continent, year)]}
\end{Highlighting}
\end{Shaded}

\begin{verbatim}
##     continent year         V1
##  1:      Asia 1952 1395357352
##  2:      Asia 1957 1562780599
##  3:      Asia 1962 1696357182
##  4:      Asia 1967 1905662900
##  5:      Asia 1972 2150972248
##  6:      Asia 1977 2384513556
##  7:      Asia 1982 2610135582
##  8:      Asia 1987 2871220762
##  9:      Asia 1992 3133292191
## 10:      Asia 1997 3383285500
## 11:      Asia 2002 3601802203
## 12:      Asia 2007 3811953827
## 13:    Europe 1952  418120846
## 14:    Europe 1957  437890351
## 15:    Europe 1962  460355155
## 16:    Europe 1967  481178958
## 17:    Europe 1972  500635059
## 18:    Europe 1977  517164531
## 19:    Europe 1982  531266901
## 20:    Europe 1987  543094160
## 21:    Europe 1992  558142797
## 22:    Europe 1997  568944148
## 23:    Europe 2002  578223869
## 24:    Europe 2007  586098529
## 25:    Africa 1952  237640501
## 26:    Africa 1957  264837738
## 27:    Africa 1962  296516865
## 28:    Africa 1967  335289489
## 29:    Africa 1972  379879541
## 30:    Africa 1977  433061021
## 31:    Africa 1982  499348587
## 32:    Africa 1987  574834110
## 33:    Africa 1992  659081517
## 34:    Africa 1997  743832984
## 35:    Africa 2002  833723916
## 36:    Africa 2007  929539692
## 37:  Americas 1952  345152446
## 38:  Americas 1957  386953916
## 39:  Americas 1962  433270254
## 40:  Americas 1967  480746623
## 41:  Americas 1972  529384210
## 42:  Americas 1977  578067699
## 43:  Americas 1982  630290920
## 44:  Americas 1987  682753971
## 45:  Americas 1992  739274104
## 46:  Americas 1997  796900410
## 47:  Americas 2002  849772762
## 48:  Americas 2007  898871184
## 49:   Oceania 1952   10686006
## 50:   Oceania 1957   11941976
## 51:   Oceania 1962   13283518
## 52:   Oceania 1967   14600414
## 53:   Oceania 1972   16106100
## 54:   Oceania 1977   17239000
## 55:   Oceania 1982   18394850
## 56:   Oceania 1987   19574415
## 57:   Oceania 1992   20919651
## 58:   Oceania 1997   22241430
## 59:   Oceania 2002   23454829
## 60:   Oceania 2007   24549947
##     continent year         V1
\end{verbatim}

As you can see, these results lend well to data we can tabulate or plot:

\begin{Shaded}
\begin{Highlighting}[]
\KeywordTok{library}\NormalTok{(}\StringTok{"gplots"}\NormalTok{)}
\KeywordTok{plot}\NormalTok{(gapminderFiveYearData[,}\KeywordTok{sum}\NormalTok{(pop), }\DataTypeTok{by=}\KeywordTok{list}\NormalTok{(continent, year)]$year,}
     \NormalTok{gapminderFiveYearData[,}\KeywordTok{sum}\NormalTok{(pop), }\DataTypeTok{by=}\KeywordTok{list}\NormalTok{(continent, year)]$V1,}
     \DataTypeTok{col=}\KeywordTok{rainbow}\NormalTok{(}\DecValTok{5}\NormalTok{)[}\KeywordTok{as.numeric}\NormalTok{(}\KeywordTok{as.factor}\NormalTok{(gapminderFiveYearData[,}\KeywordTok{sum}\NormalTok{(pop), }\DataTypeTok{by=}\KeywordTok{list}\NormalTok{(continent, year)]$continent))])}
\KeywordTok{legend}\NormalTok{(}\StringTok{"topleft"}\NormalTok{, }\DataTypeTok{fill=}\KeywordTok{rainbow}\NormalTok{(}\DecValTok{5}\NormalTok{), }\DataTypeTok{legend=}\KeywordTok{levels}\NormalTok{(}\KeywordTok{as.factor}\NormalTok{(gapminderFiveYearData[,}\KeywordTok{sum}\NormalTok{(pop), }\DataTypeTok{by=}\KeywordTok{list}\NormalTok{(continent, year)]$continent)))}
\end{Highlighting}
\end{Shaded}

\includegraphics{Figs/unnamed-chunk-32-1.pdf}\\
New and Shiny: by=.EACHI enables more explicit control of the ``by''
feature. We could manually pull out years or countries we wish to deal
with individually:

\begin{Shaded}
\begin{Highlighting}[]
\NormalTok{gapminderFiveYearData[year==}\StringTok{"1952"} \NormalTok{|}\StringTok{ }\NormalTok{year==}\StringTok{"2002"}\NormalTok{, j=}\KeywordTok{sum}\NormalTok{(pop), by=year]}
\end{Highlighting}
\end{Shaded}

\begin{verbatim}
##    year         V1
## 1: 1952 2406957151
## 2: 2002 5886977579
\end{verbatim}

\begin{Shaded}
\begin{Highlighting}[]
\NormalTok{gapminderFiveYearData[}\KeywordTok{c}\NormalTok{(}\StringTok{"New Zealand"}\NormalTok{,}\StringTok{"Australia"}\NormalTok{),}\KeywordTok{sum}\NormalTok{(gdpPercap*pop)]}
\end{Highlighting}
\end{Shaded}

\begin{verbatim}
## [1] 4.516491e+12
\end{verbatim}

\begin{Shaded}
\begin{Highlighting}[]
\NormalTok{gapminderFiveYearData[}\KeywordTok{c}\NormalTok{(}\StringTok{"New Zealand"}\NormalTok{,}\StringTok{"Australia"}\NormalTok{),}\KeywordTok{sum}\NormalTok{(gdpPercap*pop), by=year]}
\end{Highlighting}
\end{Shaded}

\begin{verbatim}
##     year           V1
##  1: 1952 108314447889
##  2: 1957 133653656027
##  3: 1962 164672906489
##  4: 1967 211917727171
##  5: 1972 268224218455
##  6: 1977 309415422324
##  7: 1982 352354302760
##  8: 1987 418903127997
##  9: 1992 472638359652
## 10: 1997 578608510367
## 11: 2002 690473760353
## 12: 2007 807314089023
\end{verbatim}

Notice in both of the above cases the countries are grouped together.
Unless specified countries will not be grouped, we can do this either
explicitly \texttt{by=country} or use the \texttt{.EACHI} options for
more complex \texttt{i} queries:

\begin{Shaded}
\begin{Highlighting}[]
\NormalTok{gapminderFiveYearData[}\KeywordTok{c}\NormalTok{(}\StringTok{"New Zealand"}\NormalTok{,}\StringTok{"Australia"}\NormalTok{),}\KeywordTok{sum}\NormalTok{(gdpPercap*pop), by=country]}
\end{Highlighting}
\end{Shaded}

\begin{verbatim}
##        country           V1
## 1: New Zealand 6.734455e+11
## 2:   Australia 3.843045e+12
\end{verbatim}

\begin{Shaded}
\begin{Highlighting}[]
\NormalTok{gapminderFiveYearData[}\KeywordTok{c}\NormalTok{(}\StringTok{"New Zealand"}\NormalTok{,}\StringTok{"Australia"}\NormalTok{),}\KeywordTok{sum}\NormalTok{(gdpPercap*pop), by=.EACHI]}
\end{Highlighting}
\end{Shaded}

\begin{verbatim}
##        country           V1
## 1: New Zealand 6.734455e+11
## 2:   Australia 3.843045e+12
\end{verbatim}

Group by multiple arguments explicitly may also give data in a more
sensible format:

\begin{Shaded}
\begin{Highlighting}[]
\NormalTok{gapminderFiveYearData[}\KeywordTok{c}\NormalTok{(}\StringTok{"New Zealand"}\NormalTok{,}\StringTok{"Australia"}\NormalTok{),}\KeywordTok{sum}\NormalTok{(gdpPercap*pop), by=}\KeywordTok{list}\NormalTok{(year, country)]}
\end{Highlighting}
\end{Shaded}

\begin{verbatim}
##     year     country           V1
##  1: 1952 New Zealand  21058193787
##  2: 1957 New Zealand  27304428858
##  3: 1962 New Zealand  32788333487
##  4: 1967 New Zealand  39459740429
##  5: 1972 New Zealand  47000447797
##  6: 1977 New Zealand  51378093149
##  7: 1982 New Zealand  56611498451
##  8: 1987 New Zealand  63050008703
##  9: 1992 New Zealand  63127124700
## 10: 1997 New Zealand  77385257446
## 11: 2002 New Zealand  90626601698
## 12: 2007 New Zealand 103655730130
## 13: 1952   Australia  87256254102
## 14: 1957   Australia 106349227169
## 15: 1962   Australia 131884573002
## 16: 1967   Australia 172457986742
## 17: 1972   Australia 221223770658
## 18: 1977   Australia 258037329175
## 19: 1982   Australia 295742804309
## 20: 1987   Australia 355853119294
## 21: 1992   Australia 409511234952
## 22: 1997   Australia 501223252921
## 23: 2002   Australia 599847158654
## 24: 2007   Australia 703658358894
##     year     country           V1
\end{verbatim}

\texttt{by=.EACHI} is a little weird, it's an explicit way of restoring
a previous version \texttt{data.table} functionality. Consider a simple
operation of counting the rows returned:

By default data.table counts all rows returned:

\begin{Shaded}
\begin{Highlighting}[]
\NormalTok{gapminderFiveYearData[}\KeywordTok{c}\NormalTok{(}\StringTok{"New Zealand"}\NormalTok{,}\StringTok{"Australia"}\NormalTok{), .N]}
\end{Highlighting}
\end{Shaded}

\begin{verbatim}
## [1] 24
\end{verbatim}

To restore previous functionality (an implicit by), \texttt{.by=.EACHI}
will count the number of rows returned \emph{for each} i. Basically
data.table was really clever and did it for you but some people took
issue with a by being performed when it wasn't specified.

\begin{Shaded}
\begin{Highlighting}[]
\NormalTok{gapminderFiveYearData[}\KeywordTok{c}\NormalTok{(}\StringTok{"New Zealand"}\NormalTok{,}\StringTok{"Australia"}\NormalTok{), .N, by=.EACHI]}
\end{Highlighting}
\end{Shaded}

\begin{verbatim}
##        country  N
## 1: New Zealand 12
## 2:   Australia 12
\end{verbatim}

\subsection{Keys}\label{keys}

\texttt{tables()} shows all tables and their SQL-like ``keys'', by
default to keys are given:

\begin{Shaded}
\begin{Highlighting}[]
\NormalTok{gapminderFiveYearData <-}\StringTok{ }\KeywordTok{fread}\NormalTok{(}\StringTok{"gapminder-FiveYearData.csv"}\NormalTok{)}
\KeywordTok{tables}\NormalTok{()}
\end{Highlighting}
\end{Shaded}

\begin{verbatim}
##      NAME                           NROW NCOL  MB
## [1,] gapminderFiveYearData         1,704    6   1
## [2,] gapminderFiveYearDataCrop     1,000    6   1
## [3,] gapminderlarge            1,656,288    6  70
## [4,] gapminderlarger           6,625,152    6 279
##      COLS                                         KEY
## [1,] country,year,pop,continent,lifeExp,gdpPercap    
## [2,] place,time,people,big place,life,money          
## [3,] country,year,pop,continent,lifeExp,gdpPercap    
## [4,] country,year,pop,continent,lifeExp,gdpPercap    
## Total: 351MB
\end{verbatim}

We can create a unique identifier as a key:

\begin{Shaded}
\begin{Highlighting}[]
\NormalTok{rowID <-}\StringTok{ }\KeywordTok{paste}\NormalTok{(gapminderFiveYearData$country, gapminderFiveYearData$year)}
\KeywordTok{head}\NormalTok{(rowID)}
\end{Highlighting}
\end{Shaded}

\begin{verbatim}
## [1] "Afghanistan 1952" "Afghanistan 1957" "Afghanistan 1962"
## [4] "Afghanistan 1967" "Afghanistan 1972" "Afghanistan 1977"
\end{verbatim}

\begin{Shaded}
\begin{Highlighting}[]
\KeywordTok{tail}\NormalTok{(}\KeywordTok{head}\NormalTok{(rowID))}
\end{Highlighting}
\end{Shaded}

\begin{verbatim}
## [1] "Afghanistan 1952" "Afghanistan 1957" "Afghanistan 1962"
## [4] "Afghanistan 1967" "Afghanistan 1972" "Afghanistan 1977"
\end{verbatim}

\begin{Shaded}
\begin{Highlighting}[]
\NormalTok{gapminderFiveYearData$rowID <-}\StringTok{ }\NormalTok{rowID}
\NormalTok{gapminderFiveYearData}
\end{Highlighting}
\end{Shaded}

\begin{verbatim}
##           country year      pop continent lifeExp gdpPercap
##    1: Afghanistan 1952  8425333      Asia  28.801  779.4453
##    2: Afghanistan 1957  9240934      Asia  30.332  820.8530
##    3: Afghanistan 1962 10267083      Asia  31.997  853.1007
##    4: Afghanistan 1967 11537966      Asia  34.020  836.1971
##    5: Afghanistan 1972 13079460      Asia  36.088  739.9811
##   ---                                                      
## 1700:    Zimbabwe 1987  9216418    Africa  62.351  706.1573
## 1701:    Zimbabwe 1992 10704340    Africa  60.377  693.4208
## 1702:    Zimbabwe 1997 11404948    Africa  46.809  792.4500
## 1703:    Zimbabwe 2002 11926563    Africa  39.989  672.0386
## 1704:    Zimbabwe 2007 12311143    Africa  43.487  469.7093
##                  rowID
##    1: Afghanistan 1952
##    2: Afghanistan 1957
##    3: Afghanistan 1962
##    4: Afghanistan 1967
##    5: Afghanistan 1972
##   ---                 
## 1700:    Zimbabwe 1987
## 1701:    Zimbabwe 1992
## 1702:    Zimbabwe 1997
## 1703:    Zimbabwe 2002
## 1704:    Zimbabwe 2007
\end{verbatim}

\begin{Shaded}
\begin{Highlighting}[]
\KeywordTok{setkey}\NormalTok{(gapminderFiveYearData, rowID)}
\KeywordTok{tables}\NormalTok{()}
\end{Highlighting}
\end{Shaded}

\begin{verbatim}
##      NAME                           NROW NCOL  MB
## [1,] gapminderFiveYearData         1,704    7   1
## [2,] gapminderFiveYearDataCrop     1,000    6   1
## [3,] gapminderlarge            1,656,288    6  70
## [4,] gapminderlarger           6,625,152    6 279
##      COLS                                               KEY  
## [1,] country,year,pop,continent,lifeExp,gdpPercap,rowID rowID
## [2,] place,time,people,big place,life,money                  
## [3,] country,year,pop,continent,lifeExp,gdpPercap            
## [4,] country,year,pop,continent,lifeExp,gdpPercap            
## Total: 351MB
\end{verbatim}

We can search rows \texttt{i} for this key:

\begin{Shaded}
\begin{Highlighting}[]
\NormalTok{gapminderFiveYearData[}\StringTok{"New Zealand 1952"}\NormalTok{,] }\CommentTok{#search row by key}
\end{Highlighting}
\end{Shaded}

\begin{verbatim}
##        country year     pop continent lifeExp gdpPercap            rowID
## 1: New Zealand 1952 1994794   Oceania   69.39  10556.58 New Zealand 1952
\end{verbatim}

In contrast to dataframes (rownames) duplicate keys are permitted:

\begin{Shaded}
\begin{Highlighting}[]
\KeywordTok{setkey}\NormalTok{(gapminderFiveYearData, country)}
\NormalTok{gapminderFiveYearData[}\StringTok{"New Zealand"}\NormalTok{,]}
\end{Highlighting}
\end{Shaded}

\begin{verbatim}
##         country year     pop continent lifeExp gdpPercap            rowID
##  1: New Zealand 1952 1994794   Oceania  69.390  10556.58 New Zealand 1952
##  2: New Zealand 1957 2229407   Oceania  70.260  12247.40 New Zealand 1957
##  3: New Zealand 1962 2488550   Oceania  71.240  13175.68 New Zealand 1962
##  4: New Zealand 1967 2728150   Oceania  71.520  14463.92 New Zealand 1967
##  5: New Zealand 1972 2929100   Oceania  71.890  16046.04 New Zealand 1972
##  6: New Zealand 1977 3164900   Oceania  72.220  16233.72 New Zealand 1977
##  7: New Zealand 1982 3210650   Oceania  73.840  17632.41 New Zealand 1982
##  8: New Zealand 1987 3317166   Oceania  74.320  19007.19 New Zealand 1987
##  9: New Zealand 1992 3437674   Oceania  76.330  18363.32 New Zealand 1992
## 10: New Zealand 1997 3676187   Oceania  77.550  21050.41 New Zealand 1997
## 11: New Zealand 2002 3908037   Oceania  79.110  23189.80 New Zealand 2002
## 12: New Zealand 2007 4115771   Oceania  80.204  25185.01 New Zealand 2007
\end{verbatim}

By default, alls rows are returned for each group (rather than only
first for dataframe), the \texttt{mult="first"} or \texttt{"last"} can
modify this:

\begin{Shaded}
\begin{Highlighting}[]
\NormalTok{gapminderFiveYearData[}\StringTok{"New Zealand"}\NormalTok{, mult=}\StringTok{"first"}\NormalTok{] }
\end{Highlighting}
\end{Shaded}

\begin{verbatim}
##        country year     pop continent lifeExp gdpPercap            rowID
## 1: New Zealand 1952 1994794   Oceania   69.39  10556.58 New Zealand 1952
\end{verbatim}

\begin{Shaded}
\begin{Highlighting}[]
\NormalTok{gapminderFiveYearData[}\StringTok{"New Zealand"}\NormalTok{, mult=}\StringTok{"last"}\NormalTok{] }
\end{Highlighting}
\end{Shaded}

\begin{verbatim}
##        country year     pop continent lifeExp gdpPercap            rowID
## 1: New Zealand 2007 4115771   Oceania  80.204  25185.01 New Zealand 2007
\end{verbatim}

Queries in data.tables aren't just \emph{easier} they're \textbf{faster}

\begin{Shaded}
\begin{Highlighting}[]
\NormalTok{gapminderFiveYearData[}\StringTok{"New Zealand"}\NormalTok{, mult=}\StringTok{"first"}\NormalTok{] }
\end{Highlighting}
\end{Shaded}

\begin{verbatim}
##        country year     pop continent lifeExp gdpPercap            rowID
## 1: New Zealand 1952 1994794   Oceania   69.39  10556.58 New Zealand 1952
\end{verbatim}

\begin{Shaded}
\begin{Highlighting}[]
\KeywordTok{system.time}\NormalTok{(gapminderFiveYearData[}\StringTok{"New Zealand"}\NormalTok{, }\DataTypeTok{mult=}\StringTok{"first"}\NormalTok{]) }\CommentTok{#time 0.001s}
\end{Highlighting}
\end{Shaded}

\begin{verbatim}
##    user  system elapsed 
##   0.000   0.000   0.001
\end{verbatim}

\begin{Shaded}
\begin{Highlighting}[]
\NormalTok{gapminderFiveYearData.dataframe <-}\StringTok{ }\KeywordTok{as.data.frame}\NormalTok{(gapminderFiveYearData)}
\NormalTok{gapminderFiveYearData.dataframe[gapminderFiveYearData.dataframe$country==}\StringTok{"New Zealand"}\NormalTok{,][}\DecValTok{1}\NormalTok{,]}
\end{Highlighting}
\end{Shaded}

\begin{verbatim}
##          country year     pop continent lifeExp gdpPercap            rowID
## 1093 New Zealand 1952 1994794   Oceania   69.39  10556.58 New Zealand 1952
\end{verbatim}

\begin{Shaded}
\begin{Highlighting}[]
\KeywordTok{system.time}\NormalTok{(gapminderFiveYearData.dataframe[gapminderFiveYearData.dataframe$country==}\StringTok{"New Zealand"}\NormalTok{,][}\DecValTok{1}\NormalTok{,])}
\end{Highlighting}
\end{Shaded}

\begin{verbatim}
##    user  system elapsed 
##       0       0       0
\end{verbatim}

Ok, that didn't seem that different. They're powerful with larger
datafiles though. Compare these examples for the same operation with
dataframes and datatables.

\begin{Shaded}
\begin{Highlighting}[]
\KeywordTok{setkey}\NormalTok{(gapminderlarger, country)}
\NormalTok{gapminderlarger[}\StringTok{"New Zealand"}\NormalTok{, mult=}\StringTok{"first"}\NormalTok{] }
\end{Highlighting}
\end{Shaded}

\begin{verbatim}
##        country year     pop continent lifeExp gdpPercap
## 1: New Zealand 1952 1994794   Oceania   69.39  10556.58
\end{verbatim}

\begin{Shaded}
\begin{Highlighting}[]
\KeywordTok{system.time}\NormalTok{(gapminderlarger[}\StringTok{"New Zealand"}\NormalTok{, }\DataTypeTok{mult=}\StringTok{"first"}\NormalTok{])}
\end{Highlighting}
\end{Shaded}

\begin{verbatim}
##    user  system elapsed 
##       0       0       0
\end{verbatim}

\begin{Shaded}
\begin{Highlighting}[]
\NormalTok{gapminderlarger.dataframe <-}\StringTok{ }\KeywordTok{as.data.frame}\NormalTok{(gapminderlarger)}
\NormalTok{gapminderlarger.dataframe[gapminderlarger.dataframe$country==}\StringTok{"New Zealand"}\NormalTok{,][}\DecValTok{1}\NormalTok{,]}
\end{Highlighting}
\end{Shaded}

\begin{verbatim}
##             country year     pop continent lifeExp gdpPercap
## 4245697 New Zealand 1952 1994794   Oceania   69.39  10556.58
\end{verbatim}

\begin{Shaded}
\begin{Highlighting}[]
\KeywordTok{system.time}\NormalTok{(gapminderlarger.dataframe[gapminderlarger.dataframe$country==}\StringTok{"New Zealand"}\NormalTok{,][}\DecValTok{1}\NormalTok{,])}
\end{Highlighting}
\end{Shaded}

\begin{verbatim}
##    user  system elapsed 
##   0.240   0.008   0.248
\end{verbatim}

Here's an example with multiple keys:

\begin{Shaded}
\begin{Highlighting}[]
\KeywordTok{setkey}\NormalTok{(gapminderlarger, country, year)}
\NormalTok{gapminderlarger[}\KeywordTok{list}\NormalTok{(}\StringTok{"New Zealand"}\NormalTok{, }\DecValTok{2007}\NormalTok{)]}
\end{Highlighting}
\end{Shaded}

\begin{verbatim}
##           country year     pop continent lifeExp gdpPercap
##    1: New Zealand 2007 4115771   Oceania  80.204  25185.01
##    2: New Zealand 2007 4115771   Oceania  80.204  25185.01
##    3: New Zealand 2007 4115771   Oceania  80.204  25185.01
##    4: New Zealand 2007 4115771   Oceania  80.204  25185.01
##    5: New Zealand 2007 4115771   Oceania  80.204  25185.01
##   ---                                                     
## 3884: New Zealand 2007 4115771   Oceania  80.204  25185.01
## 3885: New Zealand 2007 4115771   Oceania  80.204  25185.01
## 3886: New Zealand 2007 4115771   Oceania  80.204  25185.01
## 3887: New Zealand 2007 4115771   Oceania  80.204  25185.01
## 3888: New Zealand 2007 4115771   Oceania  80.204  25185.01
\end{verbatim}

\begin{Shaded}
\begin{Highlighting}[]
\KeywordTok{system.time}\NormalTok{(gapminderlarger[}\KeywordTok{list}\NormalTok{(}\StringTok{"New Zealand"}\NormalTok{, }\DecValTok{2007}\NormalTok{)])}
\end{Highlighting}
\end{Shaded}

\begin{verbatim}
##    user  system elapsed 
##   0.004   0.000   0.001
\end{verbatim}

\begin{Shaded}
\begin{Highlighting}[]
\KeywordTok{head}\NormalTok{(gapminderlarger.dataframe[gapminderlarger.dataframe$country==}\StringTok{"New Zealand"} \NormalTok{&}\StringTok{ }\NormalTok{gapminderlarger.dataframe$year==}\StringTok{"2007"}\NormalTok{,])}
\end{Highlighting}
\end{Shaded}

\begin{verbatim}
##             country year     pop continent lifeExp gdpPercap
## 4245708 New Zealand 2007 4115771   Oceania  80.204  25185.01
## 4245720 New Zealand 2007 4115771   Oceania  80.204  25185.01
## 4245732 New Zealand 2007 4115771   Oceania  80.204  25185.01
## 4245744 New Zealand 2007 4115771   Oceania  80.204  25185.01
## 4245756 New Zealand 2007 4115771   Oceania  80.204  25185.01
## 4245768 New Zealand 2007 4115771   Oceania  80.204  25185.01
\end{verbatim}

\begin{Shaded}
\begin{Highlighting}[]
\KeywordTok{system.time}\NormalTok{(gapminderlarger.dataframe[gapminderlarger.dataframe$country==}\StringTok{"New Zealand"} \NormalTok{&}\StringTok{ }\NormalTok{gapminderlarger.dataframe$year==}\StringTok{"2007"}\NormalTok{,])}
\end{Highlighting}
\end{Shaded}

\begin{verbatim}
##    user  system elapsed 
##   1.736   0.048   1.788
\end{verbatim}

\texttt{by} is faster than a simliar operation on dataframes too:

\begin{Shaded}
\begin{Highlighting}[]
\NormalTok{gapminderlarger[,}\KeywordTok{sum}\NormalTok{(gdpPercap), year]}
\end{Highlighting}
\end{Shaded}

\begin{verbatim}
##     year         V1
##  1: 1952 2056710004
##  2: 1957 2373686150
##  3: 1962 2609102091
##  4: 1967 3027502913
##  5: 1972 3737735642
##  6: 1977 4037569928
##  7: 1982 4151155538
##  8: 1987 4362066449
##  9: 1992 4504335130
## 10: 1997 5018649457
## 11: 2002 5475604411
## 12: 2007 6448520931
\end{verbatim}

\begin{Shaded}
\begin{Highlighting}[]
\KeywordTok{system.time}\NormalTok{(gapminderlarger[,}\KeywordTok{sum}\NormalTok{(gdpPercap), year])}
\end{Highlighting}
\end{Shaded}

\begin{verbatim}
##    user  system elapsed 
##   0.076   0.000   0.076
\end{verbatim}

\begin{Shaded}
\begin{Highlighting}[]
\KeywordTok{tapply}\NormalTok{(gapminderlarger.dataframe$gdpPercap,gapminderlarger.dataframe$year,sum)}
\end{Highlighting}
\end{Shaded}

\begin{verbatim}
##       1952       1957       1962       1967       1972       1977 
## 2056710004 2373686150 2609102091 3027502913 3737735642 4037569928 
##       1982       1987       1992       1997       2002       2007 
## 4151155538 4362066449 4504335130 5018649457 5475604411 6448520931
\end{verbatim}

\begin{Shaded}
\begin{Highlighting}[]
\KeywordTok{system.time}\NormalTok{(}\KeywordTok{tapply}\NormalTok{(gapminderlarger.dataframe$gdpPercap,gapminderlarger.dataframe$year,sum))}
\end{Highlighting}
\end{Shaded}

\begin{verbatim}
##    user  system elapsed 
##   0.444   0.096   0.540
\end{verbatim}

\end{document}
